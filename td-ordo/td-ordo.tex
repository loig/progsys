\documentclass[a4paper]{article}

\usepackage{fullpage}
\usepackage[T1]{fontenc}
\usepackage[french]{babel}
\usepackage[hyperfootnotes=false]{hyperref}

\usepackage{helvet}
\renewcommand{\familydefault}{\sfdefault}

\usepackage[fontsize=18pt]{scrextend}
\usepackage{setspace}
\onehalfspacing

%boxes
\newcounter{saveFootnote}
\usepackage{fancybox}
\newcommand{\important}[1]{\setcounter{saveFootnote}{\value{footnote}}\vspace{0.3cm}\doublebox{\begin{minipage}{14.5cm}#1\end{minipage}}\vspace{0.3cm}}
\newcommand{\remarque}[1]{\setcounter{saveFootnote}{\value{footnote}}\vspace{0.3cm}\fbox{\begin{minipage}{14.5cm}#1\end{minipage}}\vspace{0.3cm}}
\newcommand{\exercice}[1]{\setcounter{saveFootnote}{\value{footnote}}\vspace{0.3cm}\shadowbox{\begin{minipage}{14.5cm}#1\end{minipage}}\vspace{0.3cm}}

%code
\usepackage{listings}
\usepackage{color}
\usepackage{newfloat}
\usepackage{caption}
\DeclareFloatingEnvironment[fileext=frm,placement={htbp},name=Programme]{figprog}
\captionsetup[figprog]{labelfont=bf}
%% Golang definition for listings
%% http://github.io/julienc91/lstlistings-golang
\lstdefinelanguage{Golang}{
  morekeywords=[1]{package,import,func,type,struct,return,defer,panic,recover,select,var,const,iota,},
   morekeywords=[2]{string,uint,uint8,uint16,uint32,uint64,int,int8,int16,int32,int64,bool,float32,float64,complex64,complex128,byte,rune,uintptr, error,interface},
   morekeywords=[3]{map,slice,make,new,nil,len,cap,copy,close,true,false,delete,append,real,imag,complex,chan,},
   morekeywords=[4]{for,break,continue,range,go,goto,switch,case,fallthrough,if,
     else,default,},
   morekeywords=[5]{Println,Printf,Error,Print,Fatal,ReadFile,Open,NewScanner,Scan,Text,Err,Create,Close},
   sensitive=true,
   morecomment=[l]{//},
   morecomment=[s]{/*}{*/},
   morestring=[b]',
   morestring=[b]",
   morestring=[s]{`}{`}
   }
\lstset{
    extendedchars=\true,
    inputencoding=utf8,
    literate=%
            {é}{{\'{e}}}1
            {è}{{\`{e}}}1
            {ê}{{\^{e}}}1
            {ë}{{\¨{e}}}1
            {û}{{\^{u}}}1
            {ù}{{\`{u}}}1
            {â}{{\^{a}}}1
            {à}{{\`{a}}}1
            {î}{{\^{i}}}1
            {ô}{{\^{o}}}1
            {ç}{{\c{c}}}1
            {Ç}{{\c{C}}}1
            {É}{{\'{E}}}1
            {Ê}{{\^{E}}}1
            {À}{{\`{A}}}1
            {Â}{{\^{A}}}1
            {Î}{{\^{I}}}1,
    frame=none,
    xleftmargin=1cm,
    basicstyle=\footnotesize,
    keywordstyle=\bf\color{blue},
    numbers=none,
    numbersep=5pt,
    showstringspaces=false,
    stringstyle=\color{red},
    tabsize=4,
    language=Golang % this is it !
}
\newcommand{\sourcecode}[3]{\begin{figprog}\begin{center}\begin{minipage}{10cm}\par\noindent\rule{\textwidth}{0.4pt}\lstinputlisting{src/#1}\par\noindent\rule{\textwidth}{0.4pt}\end{minipage}\end{center}\caption{#1}\label{#3}\end{figprog}}
\newcommand{\bigsourcecode}[3]{\begin{figprog}\begin{center}\begin{minipage}{16cm}\par\noindent\rule{\textwidth}{0.4pt}\lstinputlisting{src/#1}\par\noindent\rule{\textwidth}{0.4pt}\end{minipage}\end{center}\caption{#1}\label{#3}\end{figprog}}

% macros générales
\newcommand{\file}[1]{\textsf{#1}}
\newcommand{\madoc}{\textsc{Madoc}}
\newcommand{\term}[1]{\textsf{#1}}
\newcommand{\inlinecode}[1]{\textsf{#1}}
\newcommand{\goout}[1]{\textsf{#1}} % retours de go, erreurs, etc

\title{TD3~: ordonnancement}
\author{Programmation système}
\date{BUT informatique, deuxième année}

\begin{document}

\maketitle{}

L'objectif de ce TD est de mieux comprendre les techniques d'ordonnancement vues en cours.
Pour cela, vous allez programmer chacune de ces techniques et les tester sur des ensembles de tâches.

\section{Principe général du TD}

Un code source vous est fourni (sur \madoc{}).
Vous trouverez les fichiers suivants~:
\begin{description}
  \item[fcfs.go] contient un prototype de fonction pour l'ordonnanceur FCFS (First Come First Served)
  \item[go.mod] contient la définition du module ordo, que vous allez mettre en œuvre
  \item[ordo.go] décrit la fonction main, chargée de faire appel a vos fonctions d'ordonnancement
  \item[rr.go] contient un prototype de fonction pour l'ordonnanceur RR (Round Robin)
  \item[sjf.go] contient un prototype de fonction pour l'ordonnanceur SJF (Shortest Job First)
  \item[srt.go] contient un prototype de fonction pour l'ordonnanceur SRT (Shortest Remaining Time)
  \item[static.go] contient un prototype de fonction pour l'ordonnanceur à priorités statiques
  \item[tasks.json] contient une liste de tâches au format json
\end{description}
Votre travail sera de coder les cinq fonctions définissant les ordonnanceurs vus en cours.
Ces fonctions seront ensuite appelées par le main situé dans \file{ordo.go} en fonction des arguments de la ligne de commande utilisée pour lancer le programme.

\exercice{Prenez connaissance du fichier \file{ordo.go} et essayez d'en comprendre le fonctionnement dans les grandes lignes. Quels sont les paramètres qu'accepte ce programme ? Comment faire pour utiliser l'ordonnanceur SRT sur les tâches du fichier \file{tasks2.json} ?}

Le programme, tant qu'il tourne, montre ses résultats à l'aide d'un serveur web qu'on peut trouver à l'adresse \url{http://localhost:8080}

\section{Mise en œuvre des principales techniques d'ordonnancement}

Pour chacune des cinq techniques d'ordonnancement, répondre à l'exercice suivant.

\exercice{Dessinez l'histogramme correspondant aux tâches du fichier \file{tasks.json} pour cette technique d'ordonnancement.}

\exercice{Codez l'ordonnanceur dans le fichier prévu puis testez-le sur la liste de tâches fournies et vérifiez que le résultat obtenu correspond à l'histogramme que vous aviez dessinez auparavant.}

\section{Création de listes de tâches}

\exercice{Créez au moins une nouvelle liste de tâches et testez vos ordonnanceurs dessus. Vous pouvez échanger votre liste de tâches avec vos camarades et tester vos ordonnanceurs sur leurs listes de tâches.}

\end{document}
