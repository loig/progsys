\documentclass[a4paper]{article}
\usepackage[T1]{fontenc}
\usepackage{fullpage}
\usepackage{graphicx}
\usepackage{url}

\usepackage{helvet}
\renewcommand{\familydefault}{\sfdefault}

%\usepackage[fontsize=18pt]{scrextend}
%\usepackage{setspace}
%\onehalfspacing


\newcounter{saveFootnote}
\usepackage{fancybox}
\newcommand{\exercice}[1]{\vspace{0.3cm}\shadowbox{\begin{minipage}{15cm}#1\end{minipage}}\vspace{0.3cm}}
\newcommand{\important}[1]{\setcounter{saveFootnote}{\value{footnote}}\vspace{0.3cm}\doublebox{\begin{minipage}{14.5cm}#1\end{minipage}}\vspace{0.3cm}}

\title{Réalisation du «net code» d'un jeu vidéo simple\\ Partie 1~: un premier serveur}
\author{Projet de programmation système (R3.05)\\ Année 2023-2024}
\date{loig.jezequel@univ-nantes.fr}

\begin{document}
\maketitle{}

\section{Étude du code fourni}

Avant de pouvoir commencer à travailler sur le projet, il va vous falloir comprendre, au moins dans les grandes lignes, le code qui vous est fourni. Ce code repose en grande partie sur la bibliothèque Ebitengine, que vous avez déjà eu l'occasion d'utiliser l'an dernier.

Dans l'archive on trouve les fichiers suivants :
\begin{description}
\item[main.go] qui contient la fonction \verb|main|, chargée de démarrer le jeu et des fonctions \verb|init| qui s'exécutent avant \verb|main|.
\item[game.go] qui contient la structure de données représentant le jeu ainsi qu'une méthode \verb|reset| permettant de remettre cette structure de données à zéro pour redémarrer une partie après en avoir terminé une.
\item[layout.go] qui permet de définir la taille de la fenêtre.
\item[update.go] qui contient les fonctions appelées à chaque itération de la boucle principale du jeu (pour rappel, Ebitengine appelle automatiquement la fonction \verb|Update| 60 fois par seconde).
\item[draw.go] qui contient les fonctions permettant d'afficher le jeu à l'écran (pour rappel, Ebitengine appelle automatiquement la fonction \verb|Draw| jusqu'à 60 fois par seconde).
\end{description}

\exercice{En étudiant le code, vous devez expliquer (par exemple au moyen de schémas) comment se déroule un tour de la boucle principale du jeu (i.e. un appel à Update)~: quelles sont les fonctions qui sont appelées ? à quoi servent-elles ? comment sait-on à quelle étape du jeu on est ?}

\section{Formalisation du fonctionnement du net code de base}

Pour le moment, on souhaite simplement faire en sorte que deux personnes, chacune exécutant le jeu de son côté, puissent se connecter à un même serveur et, une fois qu'elles sont toutes les deux connectées (et pas avant), puissent choisir leur couleur.

\exercice{Par un schéma, représenter la façon dont les 2 participants doivent interagir avec le serveur~: quels sont les messages qui doivent être échangés ? comment un participant peut-il savoir quand il peut passer à la sélection de son personnage ? comment le serveur peut-il savoir que tous les participants sont connectés ?}

\section{Mise en place d'un premier serveur}

On va maintenant passer à la réalisation du serveur décrit à la partie précédente.

\exercice{Écrivez le code d'un serveur capable de recevoir des demande de connexion, de détecter quand 2 clients sont connectés, puis de prévenir ces clients qu'ils sont tous là.}

\exercice{Testez votre serveur en écrivant en Go un client rudimentaire qui se connecte au serveur (et en le lançant deux fois). L'adresse du serveur devra être passée en argument aux clients.}

\section{Modification du code fourni pour interagir avec le serveur}

Finalement, on va modifier le code du jeu pour mettre en place la connexion au serveur.
On souhaite que, à l'écran d'accueil, la personne qui lance le jeu ne puisse appuyer sur entrée pour passer à la sélection de sa couleur que si 2 joueurs sont connectés.

\exercice{À quel endroit dans le code faut-il démarrer la connexion ?}

\exercice{À quel endroit faut-il attendre le message du serveur indiquant que deux clients sont connectés ?}

\exercice{Modifiez le code fourni. Testez.}

\exercice{Que se passe-t-il au niveau du jeu pendant qu'un client attend l'autre ? Expliquez. Proposez une solution.}

\exercice{Modifiez votre code pour mettre en œuvre la solution précédente afin de ne plus avoir de blocage de votre jeu quand un client attend l'autre.}

\end{document}
