\documentclass{beamer}

\setbeamertemplate{footline}[frame number]

\title{Programmation système}
\author{Présentation du cours}
\date{loig.jezequel@univ-nantes.fr}

\begin{document}

\frame{
\maketitle
}

\frame{
\frametitle{Au programme}

\begin{block}{Ce qu'on ne fera pas}
\begin{itemize}
\item Programmer notre propre système (d'exploitation)
\item De la programmation système (création de processus, etc)
\end{itemize}
\end{block}

\begin{block}{Ce qu'on fera}
\begin{itemize}
\item Focus sur la concurrence (centrale dans les OS)
\item Programmation concurrente en Go
\end{itemize}
\end{block}

\pause

\begin{center}
\alert{Les concepts abordés sont directement transposables à la programmation des systèmes d'exploitation}
\end{center}

}

\frame{
\frametitle{Organisation du cours}

\begin{block}{Cours magistraux}
3 séances, concepts de base sur la concurrence
\end{block}

\begin{block}{Travaux dirigés}
6 séances, en salle machine, la programmation concurrente en pratique
\end{block}

\begin{block}{Travaux pratiques}
10 séances (2 par semaine), projet de programmation concurrente
\end{block}

}

\frame{
\frametitle{Évaluation}

\begin{itemize}
\item DS en fin de semestre
\item Test machine en fin de semestre
\item Évaluation du projet
\end{itemize}

}

\frame{
\frametitle{Équie enseignante}

\begin{block}{Équipe}
\begin{itemize}
  \item Olivier Boutin (TD, TP)
  \item Sébastien Faucou (TD, TP)
  \item Loïg Jezequel (CM, TD, TP)
  \item Jean-François Remm (TD, TP)
\end{itemize}
\end{block}

\begin{alertblock}{Important}
N'hésitez pas à me contacter ou à contacter votre enseignant de TD si vous avez la moindre question ou le moindre soucis.
\end{alertblock}

}

\frame{
\frametitle{Pour finir}

\begin{block}{Références}
\begin{itemize}
\item Computer Systems a Programmer Perspective. Bryant, O'Hallaron.
\item Documentation officielle du langage Go
\end{itemize}
\end{block}

\pause

\begin{block}{Pratiquez}
Les concepts de ce cours seront probablement nouveaux pour vous et ne sont pas toujours évidents.
\alert{Investissez vous dans les travaux pratiques} pour mieux les comprendre.
Ne faites pas le projet au dernier moment.
\end{block}

}

\end{document}
