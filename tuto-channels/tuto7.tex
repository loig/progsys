\documentclass[a4paper]{article}

\usepackage{fullpage}
\usepackage[T1]{fontenc}
\usepackage[french]{babel}
\usepackage[hyperfootnotes=false]{hyperref}

\usepackage{helvet}
\renewcommand{\familydefault}{\sfdefault}

%\usepackage[fontsize=18pt]{scrextend}
%\usepackage{setspace}
%\onehalfspacing

%boxes
\newcounter{saveFootnote}
\usepackage{fancybox}
\newcommand{\important}[1]{\setcounter{saveFootnote}{\value{footnote}}\vspace{0.3cm}\doublebox{\begin{minipage}{14.5cm}#1\end{minipage}}\vspace{0.3cm}}
\newcommand{\remarque}[1]{\setcounter{saveFootnote}{\value{footnote}}\vspace{0.3cm}\fbox{\begin{minipage}{14.5cm}#1\end{minipage}}\vspace{0.3cm}}
\newcommand{\exercice}[1]{\setcounter{saveFootnote}{\value{footnote}}\vspace{0.3cm}\shadowbox{\begin{minipage}{14.5cm}#1\end{minipage}}\vspace{0.3cm}}

%code
\usepackage{listings}
\usepackage{color}
\usepackage{newfloat}
\usepackage{caption}
\DeclareFloatingEnvironment[fileext=frm,placement={htbp},name=Programme]{figprog}
\captionsetup[figprog]{labelfont=bf}
%% Golang definition for listings
%% http://github.io/julienc91/lstlistings-golang
\lstdefinelanguage{Golang}{
  morekeywords=[1]{package,import,func,type,struct,return,defer,panic,recover,select,var,const,iota,},
   morekeywords=[2]{string,uint,uint8,uint16,uint32,uint64,int,int8,int16,int32,int64,bool,float32,float64,complex64,complex128,byte,rune,uintptr, error,interface},
   morekeywords=[3]{map,slice,make,new,nil,len,cap,copy,close,true,false,delete,append,real,imag,complex,chan,},
   morekeywords=[4]{for,break,continue,range,go,goto,switch,case,fallthrough,if,
     else,default,},
   morekeywords=[5]{Println,Printf,Error,Print,Fatal,ReadFile,Open,NewScanner,Scan,Text,Err,Create,Close},
   sensitive=true,
   morecomment=[l]{//},
   morecomment=[s]{/*}{*/},
   morestring=[b]',
   morestring=[b]",
   morestring=[s]{`}{`}
   }
\lstset{
    extendedchars=\true,
    inputencoding=utf8,
    literate=%
            {é}{{\'{e}}}1
            {è}{{\`{e}}}1
            {ê}{{\^{e}}}1
            {ë}{{\¨{e}}}1
            {û}{{\^{u}}}1
            {ù}{{\`{u}}}1
            {â}{{\^{a}}}1
            {à}{{\`{a}}}1
            {î}{{\^{i}}}1
            {ô}{{\^{o}}}1
            {ç}{{\c{c}}}1
            {Ç}{{\c{C}}}1
            {É}{{\'{E}}}1
            {Ê}{{\^{E}}}1
            {À}{{\`{A}}}1
            {Â}{{\^{A}}}1
            {Î}{{\^{I}}}1,
    frame=none,
    xleftmargin=1cm,
    basicstyle=\footnotesize,
    keywordstyle=\bf\color{blue},
    numbers=none,
    numbersep=5pt,
    showstringspaces=false,
    stringstyle=\color{red},
    tabsize=4,
    language=Golang % this is it !
}
\newcommand{\sourcecode}[3]{\begin{figprog}\begin{center}\begin{minipage}{10cm}\par\noindent\rule{\textwidth}{0.4pt}\lstinputlisting{src/#1}\par\noindent\rule{\textwidth}{0.4pt}\end{minipage}\end{center}\caption{#1}\label{#3}\end{figprog}}
\newcommand{\bigsourcecode}[3]{\begin{figprog}\begin{center}\begin{minipage}{16cm}\par\noindent\rule{\textwidth}{0.4pt}\lstinputlisting{src/#1}\par\noindent\rule{\textwidth}{0.4pt}\end{minipage}\end{center}\caption{#1}\label{#3}\end{figprog}}

% macros générales
\newcommand{\file}[1]{\textsf{#1}}
\newcommand{\madoc}{\textsc{Madoc}}
\newcommand{\term}[1]{\textsf{#1}}
\newcommand{\inlinecode}[1]{\textsf{#1}}
\newcommand{\goout}[1]{\textsf{#1}} % retours de go, erreurs, etc

\title{Les bases du langage Go\\ Partie 7~: goroutines et canaux}
\author{Programmation système}
\date{BUT informatique, deuxième année}

\begin{document}

\maketitle{}

Comme pour le cours d'initiation au développement, le langage Go sera utilisé pour le cours de programmation système.
Ce TP a pour but de compléter les bases du langage Go étudiées l'an dernier.
On part du principe que les bases du langage Go (5 parties) vues en initiation au développement sont maîtrisées.

\important{Dans tout ce TP un texte avec cet encadrement est un texte contenant une information importante, il faut donc le lire avec attention et veiller à en tenir compte.}

\remarque{Un texte avec cet encadrement est une remarque.}

\exercice{Un texte avec cet encadrement est un travail que vous avez à faire.}


\remarque{Si vous souhaitez avoir un autre point de vue sur le contenu de ce TP, vous pouvez consulter, par exemple, le site \emph{Go by Example}\footnotemark{}.}
\stepcounter{saveFootnote}
\footnotetext[\value{saveFootnote}]{\url{https://gobyexample.com/}}

\newpage{}

\section{Goroutines}

En Go, une goroutine est un moyen d'exécuter une fonction de manière concurrente par rapport au reste du programme.
On dispose d'un seul moyen de contrôle sur les goroutines~: le mot clé \inlinecode{go} qui permet de les créer.

Le programme~\ref{code:goroutines} montre la création d'une goroutine à partir de l'appel de fonction \inlinecode{iter(1)}.

\sourcecode{goroutines.go}{Démarrer des goroutines}{code:goroutines}

\exercice{Récupérez le fichier \file{goroutines.go} sur \madoc{} et testez le programme plusieurs fois. Obtenez-vous toujours le même résultat ? Pourquoi ? Est-ce que chacune des deux goroutines affiche bien 100 messages à chaque fois ? Pourquoi ? Que se passe-t-il si on inverse les deux lignes du \inlinecode{main} ? Expliquez.}

\exercice{Modifiez le fichier \file{goroutines.go} pour lancer 100 goroutines (numérotées de 1 à 100) au lieu de 2.}


\section{Canaux}
% créer un canal
% écrire dans un canal
% lire dans un canal
% canaux unidirectionnels ?

En Go, la mémoire est partagée entre goroutines, on peut donc, comme dans le programme~\ref{code:race} avoir plusieurs goroutines qui utilisent une même variable pour communiquer.

\sourcecode{partage.go}{Mémoire partagée}{code:race}

\exercice{Récupérez le fichier \file{partage.go} sur \madoc{} et testez-le.}

Utiliser la mémoire partagée pour communiquer entre goroutines n'est pas recommandé~: on peut avoir des conflits entre les accès aux variables partagées par les différentes goroutines.
Le mécanisme recommandé en Go pour la communication entre goroutines s'appelle les canaux.

Le programme~\ref{code:chan} montre comment créer un canal contenant des entiers et de taille 3 (c'est-à-dire pouvant contenir 3 valeurs en même temps).
On écrit ensuite des valeurs dans le canal, puis on en lit (et on les affiche).

\sourcecode{chan.go}{Utilisation des canaux}{code:chan}

\exercice{Récupérez le fichier \file{chan.go} sur \madoc{} et testez-le. Dans quel ordre sont rangés les messages dans un canal ?}

\remarque{On parle de deadlock lorsque chaque goroutine d'un programme est en attente sur une opération nécessitant l'intervention d'une autre goroutine pour être débloquée.}

\exercice{Modifiez le programme \file{chan.go} pour ajouter une écriture supplémentaire dans \inlinecode{c} avant les lectures. Testez votre programme. Qu'en déduisez-vous ?}

\exercice{Modifiez le programme \file{chan.go} pour ajouter deux lectures supplémentaires dans \inlinecode{c} après les écritures. Testez votre programme. Qu'en déduisez-vous ?}

On peut bien sûr créer des canaux contenant d'autres types de valeurs que des entiers : booléens, chaînes de caractères, tableaux, ou même canaux.

\exercice{Réécrivez le programme~\ref{code:race} en remplaçant l'utilisation de variables partagées par l'utilisation de canaux.}

On peut aussi créer des canaux de taille 0 en ne précisant pas de taille dans \inlinecode{make}, comme dans le programme~\ref{code:zero}.
Une écriture dans un tel canal n'est alors possible que si une lecture à lieu au même moment (et réciproquement).

\sourcecode{zero.go}{Canal de taille 0}{code:zero}

\exercice{Récupérez le ficher \file{zero.go} sur \madoc{} et testez-le. Modifiez-le pour donner une grande taille au canal \inlinecode{c} lors de sa création. Testez à nouveau. Quelle différence observez-vous ? Expliquez.}

\section{La structure de contrôle select}
% select
% canal du temps écoulé (time)
% exo : boucle ou chaque itération dure au max x secondes
% exo : boucle ou somme des temps des itérations ne dépasse pas x secondes

Les canaux viennent avec une structure de contrôle permettant de faire une attente conditionnelle~: \inlinecode{select}.
Un exemple d'utilisation de cette structure de contrôle est donné dans le programme~\ref{code:select}.

\sourcecode{select.go}{La structure de contrôle select}{code:select}

\exercice{Expliquez en détails le fonctionnement du programme~\ref{code:select}.}

\exercice{Récupérez le fichier \file{select.go} sur \madoc{} et testez-le plusieurs fois.}

\exercice{Renseignez-vous sur la fonction \inlinecode{After} du paquet \inlinecode{time}. Modifiez \file{select.go} en ajoutant une condition dans le \inlinecode{select} pour limiter l'attente à 500ms.}

\end{document}
