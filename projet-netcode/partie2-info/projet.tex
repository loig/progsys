\documentclass[a4paper]{article}
\usepackage[T1]{fontenc}
\usepackage{fullpage}
\usepackage{graphicx}
\usepackage{url}

\usepackage{helvet}
\renewcommand{\familydefault}{\sfdefault}

\usepackage[fontsize=18pt]{scrextend}
\usepackage{setspace}
\onehalfspacing

\usepackage{fancybox}
\newcommand{\exercice}[1]{\vspace{0.3cm}\shadowbox{\begin{minipage}{15cm}#1\end{minipage}}\vspace{0.3cm}}

\title{Réalisation du «net code» d'un jeu vidéo simple\\ Partie 2~: transmettre de l'information aux clients}
\author{Projet de programmation système (R3.05)\\ Année 2022-2023}
\date{loig.jezequel@univ-nantes.fr}

\begin{document}
\maketitle{}

Quand vous commencez ce TP vous avez normalement mis en place un serveur capable de signaler aux clients quand ils sont quatre à s'être connectés.

\section{Un peu plus loin dans le jeu}

C'est votre serveur qui va contrôler l'avancement du jeu pour les clients~: il leur permettra, à chaque phase du jeu (connexion, choix des personnages, décompte, course, score) de savoir quand les autres sont prêts à passer à la suite.

Cela se déroulera à peu près toujours de la même façon, que vous avez déjà mise en œuvre pour détecter la connexion des quatre clients~: le serveur attend quatre messages (un de chaque client) puis prévient les clients qu'ils peuvent passer à la phase suivante.

\exercice{Étendez votre serveur pour prendre en compte la sélection des personnages. Pour le moment on veut juste savoir quand tous les joueurs ont sélectionné leur personnage, on ne souhaite pas (encore) transmettre cette information aux autres. Vous pourrez représenter les interactions entre clients et serveur par un schéma si cela vous aide.}

\exercice{Est-il nécessaire d'ajouter une synchronisation supplémentaire pour gérer le décompte avant la course ? Si oui, faites-le.}

\section{Transmettre des informations aux clients}

Pour le moment, votre serveur n'est pas capable de transmettre d'informations précises aux clients, il ne fait que leur signaler quand ils peuvent changer de phase.

Cependant, à la fin de la course, il faut que chaque client puisse récupérer les temps des autres afin de déterminer le classement et de l'afficher.

\exercice{Représentez les échanges qui doivent avoir lieu entre clients et serveur pour que tout le monde obtienne les temps des autres. Proposez un protocole, c'est-à-dire un format de messages, pour mettre cela en œuvre.}

\exercice{Étendez votre serveur pour prendre en compte la fin de la course. Chaque client doit afficher tous les temps, dans l'ordre d'arrivée de la course. Les écrans finaux de tous les clients doivent donc afficher le même classement et les mêmes temps en fin de course.}

\section{La fin du jeu}

Pour mettre en place tout le flot du jeu il ne reste plus qu'à permettre aux joueurs de se synchroniser pour redémarrer une course.
Quand tous les joueurs ont appuyé sur espace, une nouvelle course démarre.

\exercice{Faites un schéma pour représenter les échanges entre clients et serveur tout au long du jeu, en incluant cette dernière synchronisation.}

\exercice{Ajouter cette dernière synchronisation à votre jeu. N'oubliez pas de tester en affrontant vos camarades.}

\section{Première amélioration}

\exercice{Affichez à l'écran le nombre de joueurs connectés (en début de jeu) et le nombre de joueurs prêts à redémarrer (à la fin d'une course). Ces nombres doivent évoluer dynamiquement dès que le serveur obtient de nouvelles informations. Au besoin, représenter sur un schéma les interactions entre serveur et clients.}


\end{document}
