\documentclass[a4paper]{article}

\usepackage{fullpage}
\usepackage[T1]{fontenc}
\usepackage[french]{babel}
\usepackage[hyperfootnotes=false]{hyperref}

\usepackage{helvet}
\renewcommand{\familydefault}{\sfdefault}

%\usepackage[fontsize=18pt]{scrextend}
%\usepackage{setspace}
%\onehalfspacing

%sections
\usepackage{titlesec}
\titleformat{\section}{\normalfont\Large\bfseries}{\setcounter{numex}{0}Exercice~\arabic{section}.}{1em}{}

%boxes
\newcounter{numex}
\usepackage{fancybox}
\newcommand{\exercice}[1]{\stepcounter{numex}\vspace{0.3cm}\shadowbox{\begin{minipage}{14.5cm}{\bf Question \arabic{section}.\arabic{numex}.} #1\end{minipage}}\vspace{0.3cm}}
\newcommand{\important}[1]{\vspace{0.3cm}\doublebox{\begin{minipage}{14.5cm}#1\end{minipage}}\vspace{0.3cm}}

%code
\usepackage{listings}
\usepackage{color}
\usepackage{newfloat}
\usepackage{caption}
\DeclareFloatingEnvironment[fileext=frm,placement={htbp},name=Programme]{figprog}
\captionsetup[figprog]{labelfont=bf}
%% Golang definition for listings
%% http://github.io/julienc91/lstlistings-golang
\lstdefinelanguage{Golang}{
  morekeywords=[1]{package,import,func,type,struct,return,defer,panic,recover,select,var,const,iota,},
   morekeywords=[2]{string,uint,uint8,uint16,uint32,uint64,int,int8,int16,int32,int64,bool,float32,float64,complex64,complex128,byte,rune,uintptr, error,interface},
   morekeywords=[3]{map,slice,make,new,nil,len,cap,copy,close,true,false,delete,append,real,imag,complex,chan,},
   morekeywords=[4]{for,break,continue,range,go,goto,switch,case,fallthrough,if,
     else,default,},
   morekeywords=[5]{Println,Printf,Error,Print,Fatal,ReadFile,Open,NewScanner,Scan,Text,Err,Create,Close},
   sensitive=true,
   morecomment=[l]{//},
   morecomment=[s]{/*}{*/},
   morestring=[b]',
   morestring=[b]",
   morestring=[s]{`}{`}
   }
\lstset{
    extendedchars=\true,
    inputencoding=utf8,
    literate=%
            {é}{{\'{e}}}1
            {è}{{\`{e}}}1
            {ê}{{\^{e}}}1
            {ë}{{\¨{e}}}1
            {û}{{\^{u}}}1
            {ù}{{\`{u}}}1
            {â}{{\^{a}}}1
            {à}{{\`{a}}}1
            {î}{{\^{i}}}1
            {ô}{{\^{o}}}1
            {ç}{{\c{c}}}1
            {Ç}{{\c{C}}}1
            {É}{{\'{E}}}1
            {Ê}{{\^{E}}}1
            {À}{{\`{A}}}1
            {Â}{{\^{A}}}1
            {Î}{{\^{I}}}1,
    frame=none,
    xleftmargin=1cm,
    basicstyle=\footnotesize,
    keywordstyle=\bf\color{blue},
    numbers=none,
    numbersep=5pt,
    showstringspaces=false,
    stringstyle=\color{red},
    tabsize=4,
    language=Golang % this is it !
}
\newcommand{\sourcecode}[3]{\begin{figprog}\begin{center}\begin{minipage}{10cm}\par\noindent\rule{\textwidth}{0.4pt}\lstinputlisting{src/#1}\par\noindent\rule{\textwidth}{0.4pt}\end{minipage}\end{center}\caption{#1}\label{#3}\end{figprog}}
\newcommand{\bigsourcecode}[3]{\begin{figprog}\begin{center}\begin{minipage}{16cm}\par\noindent\rule{\textwidth}{0.4pt}\lstinputlisting{src/#1}\par\noindent\rule{\textwidth}{0.4pt}\end{minipage}\end{center}\caption{#1}\label{#3}\end{figprog}}

% macros générales
\newcommand{\file}[1]{\textsf{#1}}
\newcommand{\madoc}{\textsc{Madoc}}
\newcommand{\term}[1]{\textsf{#1}}
\newcommand{\inlinecode}[1]{\textsf{#1}}
\newcommand{\goout}[1]{\textsf{#1}} % retours de go, erreurs, etc

\title{Test machine 2022/2023}
\author{Programmation système}
\date{BUT informatique, deuxième année}

\begin{document}

\maketitle{}

{\bf Vous disposez de 1h} pour faire ce test. À la fin du temps, vous devrez {\bf déposer tous les fichiers} à rendre (code Go), ainsi qu'un fichier pdf contenant vos réponses aux questions qui ne demandent pas de coder, {\bf dans l'espace de rendu à votre nom} (situé dans le dossier \file{Étudiants} du devoir).

Les deux exercices sont indépendants.
Chacun rapportera la moitié des points (10/20).
Il est conseillé de faire les questions de chaque exercice dans l'ordre, elles sont faites pour vous guider.

\important{Dans tout ce test, avoir une \emph{race-condition} ou un \emph{deadlock} dans votre code entraînera la perte de tous les points de la question.}

\important{À toute fin utile, on rappelle qu'il est possible d'obtenir la documentation sur un paquet XXX en utilisant la commande suivante dans un terminal~: \inlinecode{go doc XXX} et la documentation sur la fonction YYY de ce paquet en utilisant la commande~: \inlinecode{go doc XXX.YYY}}

\section{Producteur/consommateur}

Le dossier \file{ex1} fourni avec cet énoncé contient du code source Go.
Un fichier \file{main.go} contient le code de la fonction \inlinecode{main}, qui se charge de démarrer deux goroutines~: \inlinecode{producer} et \inlinecode{consumer}.
Un fichier \file{prod.go} contient le code de la fonction \inlinecode{producer} et un fichier \file{cons.go} contient un prototype de la fonction \inlinecode{consumer}.

\exercice{Expliquer l'intérêt de lancer la fonction \inlinecode{producer} sans utiliser le mot clé \inlinecode{go}.}

\exercice{Expliquer en quelques mots ce que fait la fonction \inlinecode{producer}.}

\exercice{Coder la fonction \inlinecode{consumer} pour faire en sorte qu'elle affiche dans le terminal tout ce que la fonction \inlinecode{producer} écrit dans son canal. Vous rendrez ce code sous le nom \file{consv1.go}.}

\exercice{Coder la fonction \inlinecode{consumer} pour faire en sorte qu'elle affiche dans le terminal tout ce que la fonction \inlinecode{producer} écrit dans son canal mais en étant régulière : un nombre par seconde. Si jamais, au moment où un affichage doit avoir lieu, le canal ne contient rien, vous afficherez \emph{trop tard} à la place. Vous pourrez utiliser pour cela la fonction \inlinecode{time.After()} du packet \inlinecode{time}. Vous rendrez ce code sous le nom \file{consv2.go}.}

\exercice{Coder la fonction \inlinecode{consumer} pour faire en sorte qu'elle affiche dans le terminal uniquement les nombres impairs que la fonction \inlinecode{producer} écrit dans son canal et en étant régulière : un nombre par seconde. Si jamais, au moment où un affichage doit avoir lieu, le canal ne contient rien, vous afficherez \emph{trop tard} à la place. Vous pourrez utiliser pour cela la fonction \inlinecode{time.After()} du packet \inlinecode{time}. Vous rendrez ce code sous le nom \file{consv3.go}.}

\section{Client/serveur}

Le dossier \file{ex2} fourni avec cet énoncé contient du code source Go~: dans un dossier \file{server} vous trouverez le code d'un serveur très simple.

\exercice{Sur quel port le serveur codé dans \file{server/server.go} écoute-t-il ?}

\exercice{Corriger le code du fichier \file{server/server.go} de façon à traiter correctement toutes les erreurs potentielles déclenchées par les fonctions d'interaction avec le réseau (Listen, Accept, Read, Write, etc). En dehors de cette prise en compte des erreurs, vous ne devez pas modifier le code. Vous rendrez ceci sous la forme d'un fichier \file{serverv2.go}.}

\exercice{Expliquer le principe général du modèle client/serveur en quelques mots (et éventuellement en vous appuyant sur un schéma).}

\exercice{Représenter graphiquement les messages qui devraient être échangés entre le client et le serveur pour faire afficher le message \emph{Bravo !} au serveur codé dans \file{server/server.go} dans le cas où toutes les valeurs tirées aléatoirement valent 0 (donc, à l'étape 2, \inlinecode{req} vaut 5 et, à l'étape 3, \inlinecode{turn} vaut 3).}

\exercice{En vous basant sur la représentation graphique précédente des échanges entre client et serveur, codez un client qui fait afficher \emph{Bravo !} au serveur codé dans \file{server/server.go} quelles que soient les valeurs tirées aléatoirement. Vous pouvez vous appuyer sur votre version du serveur (avec la gestion des erreurs) pour faire vos tests, mais le client doit fonctionner avec la version d'origine du serveur qui vous a été distribuée au début du test. Vous rendrez votre code sous la forme d'un fichier \file{client.go}.}
% je donne un code de serveur, il faut écrire un client qui répond au protocole mis en place


\end{document}
