\documentclass[a4paper]{article}

\usepackage{fullpage}
\usepackage[T1]{fontenc}
\usepackage[french]{babel}
\usepackage[hyperfootnotes=false]{hyperref}

\usepackage{helvet}
\renewcommand{\familydefault}{\sfdefault}

%\usepackage[fontsize=18pt]{scrextend}
%\usepackage{setspace}
%\onehalfspacing

%sections
\usepackage{titlesec}
\titleformat{\section}{\normalfont\Large\bfseries}{\setcounter{numex}{0}Exercice~\arabic{section}.}{1em}{}

%boxes
\newcounter{numex}
\usepackage{fancybox}
\newcommand{\exercice}[1]{\stepcounter{numex}\vspace{0.3cm}\shadowbox{\begin{minipage}{14.5cm}{\bf Question \arabic{section}.\arabic{numex}.} #1\end{minipage}}\vspace{0.3cm}}
\newcommand{\important}[1]{\vspace{0.3cm}\doublebox{\begin{minipage}{14.5cm}#1\end{minipage}}\vspace{0.3cm}}

%code
\usepackage{listings}
\usepackage{color}
\usepackage{newfloat}
\usepackage{caption}
\DeclareFloatingEnvironment[fileext=frm,placement={htbp},name=Programme]{figprog}
%\captionsetup[figprog]{labelfont=bf}
%% Golang definition for listings
%% http://github.io/julienc91/lstlistings-golang
\lstdefinelanguage{Golang}{
  morekeywords=[1]{package,import,func,type,struct,return,defer,panic,recover,select,var,const,iota,},
   morekeywords=[2]{string,uint,uint8,uint16,uint32,uint64,int,int8,int16,int32,int64,bool,float32,float64,complex64,complex128,byte,rune,uintptr, error,interface},
   morekeywords=[3]{map,slice,make,new,nil,len,cap,copy,close,true,false,delete,append,real,imag,complex,chan,},
   morekeywords=[4]{for,break,continue,range,go,goto,switch,case,fallthrough,if,
     else,default,},
   morekeywords=[5]{Println,Printf,Error,Print,Fatal,ReadFile,Open,NewScanner,Scan,Text,Err,Create,Close},
   sensitive=true,
   morecomment=[l]{//},
   morecomment=[s]{/*}{*/},
   morestring=[b]',
   morestring=[b]",
   morestring=[s]{`}{`}
   }
\lstset{
    extendedchars=\true,
    inputencoding=utf8,
    literate=%
            {é}{{\'{e}}}1
            {è}{{\`{e}}}1
            {ê}{{\^{e}}}1
            {ë}{{\¨{e}}}1
            {û}{{\^{u}}}1
            {ù}{{\`{u}}}1
            {â}{{\^{a}}}1
            {à}{{\`{a}}}1
            {î}{{\^{i}}}1
            {ô}{{\^{o}}}1
            {ç}{{\c{c}}}1
            {Ç}{{\c{C}}}1
            {É}{{\'{E}}}1
            {Ê}{{\^{E}}}1
            {À}{{\`{A}}}1
            {Â}{{\^{A}}}1
            {Î}{{\^{I}}}1,
    frame=none,
    xleftmargin=1cm,
    basicstyle=\footnotesize,
    keywordstyle=\bf\color{blue},
    numbers=none,
    numbersep=5pt,
    showstringspaces=false,
    stringstyle=\color{red},
    tabsize=4,
    language=Golang % this is it !
}
\newcommand{\sourcecode}[3]{\begin{figprog}\begin{center}\begin{minipage}{10cm}\par\noindent\rule{\textwidth}{0.4pt}\lstinputlisting{src/#1}\par\noindent\rule{\textwidth}{0.4pt}\end{minipage}\end{center}\caption{#1}\label{#3}\end{figprog}}
\newcommand{\bigsourcecode}[3]{\begin{figprog}\begin{center}\begin{minipage}{16cm}\par\noindent\rule{\textwidth}{0.4pt}\lstinputlisting{src/#1}\par\noindent\rule{\textwidth}{0.4pt}\end{minipage}\end{center}\caption{#1}\label{#3}\end{figprog}}

% macros générales
\newcommand{\file}[1]{\textsf{#1}}
\newcommand{\madoc}{\textsc{Madoc}}
\newcommand{\term}[1]{\textsf{#1}}
\newcommand{\inlinecode}[1]{\textsf{#1}}
\newcommand{\goout}[1]{\textsf{#1}} % retours de go, erreurs, etc

\title{Évaluation du projet 2024/2025}
\author{Programmation système}
\date{BUT informatique, deuxième année}

\begin{document}

\maketitle{}

{\bf Vous disposez de 1h} pour faire ce test. À la fin du temps, vous devrez {\bf déposer une archive dans l'espace de rendu à votre nom} (situé dans le dossier \file{Étudiants} du devoir).
Cette archive doit contenir~:
\begin{itemize}
\item tous les fichiers à rendre (code Go), et
\item un fichier pdf avec vos réponses aux questions qui ne demandent pas de coder
\end{itemize}.

Il est conseillé de faire les questions dans l'ordre, elles sont faites pour vous guider.

\important{Dans tout ce test, avoir une \emph{race-condition} ou un \emph{deadlock} dans votre code entraînera la perte de tous les points de la question.}

\important{À toute fin utile, on rappelle qu'il est possible d'obtenir la documentation sur un paquet XXX en utilisant la commande suivante dans un terminal~: \inlinecode{go doc XXX} et la documentation sur la fonction YYY de ce paquet en utilisant la commande~: \inlinecode{go doc XXX.YYY}}

Le dossier \file{projet} fourni avec cet énoncé contient du code source Go~: dans un dossier \file{client} vous trouverez le code d'un jeu basique (deviner un nombre entre 0 et 99 en proposant des valeurs les unes après les autres jusqu'à trouver le bon nombre) et dans le dossier \file{serveur} vous trouverez le code d'un serveur très rudimentaire.

\vfill

\section{Serveur}

Dans cette partie on s'intéresse au code du serveur situé dans le dossier \file{serveur} (donc, on ne s'intéresse pas au code situé dans le dossier \file{client}, il n'est pas nécessaire d'avoir regardé ce code pour faire cette partie).

\exercice{Sur quel port le serveur codé dans \file{serveur/main.go} écoute-t-il ?}

\exercice{Expliquer le principe général du modèle client/serveur en quelques mots (et éventuellement en vous appuyant sur un schéma).}

\exercice{Représenter graphiquement une séquence de messages qui pourraient être échangés entre un client et le serveur pour faire afficher le message \emph{C'est fini, bravo !} au serveur (ligne 85 dans \file{serveur/main.go}) dans le cas où answer vaut 73. Faites en sorte qu'au cours de cette séquence le serveur envoie au moins une fois chaque message différent qu'il peut envoyer.}

\newpage{}

\section{Client}

On s'intéresse maintenent au code du client, situé dans le dossier \file{client}.

Le dossier \file{client} contient deux fichiers de code source go~\file{main.go} et \file{pastouche.go}.
Comme son nom l'indique, il ne faut pas modifier le deuxième fichier (et, normalement, vous n'avez même pas besoin de l'ouvrir).

Ces deux fichiers sont le code d'un jeu très simple où il faut deviner un nombre entre 0 et 99 en faisant des essais successifs.
À chaque proposition de nombre une réponse indique si la proposition était un nombre trop grand, trop petit ou le bon nombre.

Le code du jeu repose sur les mêmes principes que le code de votre projet.
On n'ira pas tout modifier dans ce devoir mais on s'intéressera uniquement à trois fonctions~: \file{main}, \file{sendGuess} et \file{checkGuess}.
De plus, on devra manipuler le champs state de la structure game (que vous n'avez pas besoin d'aller regarder) pour lui donner l'une des trois valeurs possibles~: \file{stateGuess} (quand le joueur doit proposer un nombre), \file{stateRes} quand on est en train de comparer le nombre proposé avec le nombre à deviner et \file{stateWon} quand le joueur a trouvé le nombre à deviner.

\important{Attention, dans le jeu que vous rendez il ne faut jamais que la petite animation en bas à droite de la fenêtre puisse s'arrête de tourner, c'est-à-dire que les communications avec le réseau ne doivent jamais ralentir le jeu. Pour cela, vous mettrez en œuvre les stratégies vues en projet pour envoyer et recevoir des messages.}

\important{Si vous souhaitez avoir des variables partagées entre plusieurs fonctions utilisez des variables globales, ne modifiez pas le fichier \file{pastouche.go} (même si c'est vrai que c'est plus propre d'ajouter des champs à la structure game).}

\exercice{Modifier la fonction \file{main} pour établir une connexion avec le serveur.}

Quand on est dans l'état \file{stateGuess} on peut utiliser les boutons du jeu pour rentrer un nombre.
Au moment où on appuie sur le bouton \file{Proposer} la fonction \file{sendGuess} est appelée.

\exercice{Àjoutez dans la fonction \file{sendGuess} l'envoie d'un message (dont vous déterminerez le contenu à partir de vos réponses à l'exercice précédent) au serveur quand la fonction est appelée.}

Quand on est dans l'état \file{stateRes} la fonction \file{checkGuess} est appelée à chaque mise-à-jour du jeu (donc 60 fois par seconde).

\exercice{Modifier la fonction \file{checkGuess}  pour recevoir un message du serveur et utiliser son contenu pour déterminer si la dernière proposition faite était bonne, trop petite ou trop grande (après ça {\bf la variable answer ne sert donc plus à rien, vous pouvez la supprimer}).}

Quand on est dans l'état \file{stateWon} le jeu est terminé et plus rien ne se passe.

\exercice{Faites les dernières modifications à votre code pour que le serveur arrive à la fin de son exécution (affiche \emph{C'est fini, bravo !}) quand le joueur a trouvé le bon nombre.}

\end{document}
