\documentclass[a4paper]{article}

\usepackage{fullpage}
\usepackage[T1]{fontenc}
\usepackage[french]{babel}
\usepackage[hyperfootnotes=false]{hyperref}

\usepackage{helvet}
\renewcommand{\familydefault}{\sfdefault}

%\usepackage[fontsize=18pt]{scrextend}
%\usepackage{setspace}
%\onehalfspacing

%sections
\usepackage{titlesec}
\titleformat{\section}{\normalfont\Large\bfseries}{\setcounter{numex}{0}Exercice~\arabic{section}.}{1em}{}

%boxes
\newcounter{numex}
\usepackage{fancybox}
\newcommand{\exercice}[1]{\stepcounter{numex}\vspace{0.3cm}\shadowbox{\begin{minipage}{14.5cm}{\bf Question \arabic{section}.\arabic{numex}.} #1\end{minipage}}\vspace{0.3cm}}
\newcommand{\important}[1]{\vspace{0.3cm}\doublebox{\begin{minipage}{14.5cm}#1\end{minipage}}\vspace{0.3cm}}

%code
\usepackage{listings}
\usepackage{color}
\usepackage{newfloat}
\usepackage{caption}
\DeclareFloatingEnvironment[fileext=frm,placement={htbp},name=Programme]{figprog}
%\captionsetup[figprog]{labelfont=bf}
%% Golang definition for listings
%% http://github.io/julienc91/lstlistings-golang
\lstdefinelanguage{Golang}{
  morekeywords=[1]{package,import,func,type,struct,return,defer,panic,recover,select,var,const,iota,},
   morekeywords=[2]{string,uint,uint8,uint16,uint32,uint64,int,int8,int16,int32,int64,bool,float32,float64,complex64,complex128,byte,rune,uintptr, error,interface},
   morekeywords=[3]{map,slice,make,new,nil,len,cap,copy,close,true,false,delete,append,real,imag,complex,chan,},
   morekeywords=[4]{for,break,continue,range,go,goto,switch,case,fallthrough,if,
     else,default,},
   morekeywords=[5]{Println,Printf,Error,Print,Fatal,ReadFile,Open,NewScanner,Scan,Text,Err,Create,Close},
   sensitive=true,
   morecomment=[l]{//},
   morecomment=[s]{/*}{*/},
   morestring=[b]',
   morestring=[b]",
   morestring=[s]{`}{`}
   }
\lstset{
    extendedchars=\true,
    inputencoding=utf8,
    literate=%
            {é}{{\'{e}}}1
            {è}{{\`{e}}}1
            {ê}{{\^{e}}}1
            {ë}{{\¨{e}}}1
            {û}{{\^{u}}}1
            {ù}{{\`{u}}}1
            {â}{{\^{a}}}1
            {à}{{\`{a}}}1
            {î}{{\^{i}}}1
            {ô}{{\^{o}}}1
            {ç}{{\c{c}}}1
            {Ç}{{\c{C}}}1
            {É}{{\'{E}}}1
            {Ê}{{\^{E}}}1
            {À}{{\`{A}}}1
            {Â}{{\^{A}}}1
            {Î}{{\^{I}}}1,
    frame=none,
    xleftmargin=1cm,
    basicstyle=\footnotesize,
    keywordstyle=\bf\color{blue},
    numbers=none,
    numbersep=5pt,
    showstringspaces=false,
    stringstyle=\color{red},
    tabsize=4,
    language=Golang % this is it !
}
\newcommand{\sourcecode}[3]{\begin{figprog}\begin{center}\begin{minipage}{10cm}\par\noindent\rule{\textwidth}{0.4pt}\lstinputlisting{src/#1}\par\noindent\rule{\textwidth}{0.4pt}\end{minipage}\end{center}\caption{#1}\label{#3}\end{figprog}}
\newcommand{\bigsourcecode}[3]{\begin{figprog}\begin{center}\begin{minipage}{16cm}\par\noindent\rule{\textwidth}{0.4pt}\lstinputlisting{src/#1}\par\noindent\rule{\textwidth}{0.4pt}\end{minipage}\end{center}\caption{#1}\label{#3}\end{figprog}}

% macros générales
\newcommand{\file}[1]{\textsf{#1}}
\newcommand{\madoc}{\textsc{Madoc}}
\newcommand{\term}[1]{\textsf{#1}}
\newcommand{\inlinecode}[1]{\textsf{#1}}
\newcommand{\goout}[1]{\textsf{#1}} % retours de go, erreurs, etc

\title{Test machine 2023/2024}
\author{Programmation système}
\date{BUT informatique, deuxième année}

\begin{document}

\maketitle{}

{\bf Vous disposez de 1h} pour faire ce test. À la fin du temps, vous devrez {\bf déposer tous les fichiers} à rendre (code Go), ainsi qu'un fichier pdf contenant vos réponses aux questions qui ne demandent pas de coder, {\bf dans l'espace de rendu à votre nom} (situé dans le dossier \file{Étudiants} du devoir).

Les deux exercices sont indépendants.
Chacun rapportera la moitié des points (10/20).
Il est conseillé de faire les questions de chaque exercice dans l'ordre, elles sont faites pour vous guider.

\important{Dans tout ce test, avoir une \emph{race-condition} ou un \emph{deadlock} dans votre code entraînera la perte de tous les points de la question.}

\important{À toute fin utile, on rappelle qu'il est possible d'obtenir la documentation sur un paquet XXX en utilisant la commande suivante dans un terminal~: \inlinecode{go doc XXX} et la documentation sur la fonction YYY de ce paquet en utilisant la commande~: \inlinecode{go doc XXX.YYY}}

\section{Ordonnancement}

Le dossier \file{ex1} fourni avec cet énoncé contient du code source Go.
Un fichier \file{main.go} contient le code de la fonction \inlinecode{main}, qui se charge de démarrer deux goroutines~: \inlinecode{generator} et \inlinecode{scheduler}.
Un fichier \file{gen.go} contient le code de la fonction \inlinecode{generator} et un fichier \file{sched.go} contient un prototype de la fonction \inlinecode{scheduler}.

\exercice{Expliquer l'intérêt de lancer la fonction \inlinecode{generator} sans utiliser le mot clé \inlinecode{go}.}

\exercice{Coder la fonction \inlinecode{scheduler} pour faire en sorte qu'elle affiche dans le terminal tout ce qu'elle reçoit dans son canal. Vous rendrez ce code sous le nom \file{schedv1.go}.}

\exercice{Expliquer en quelques mots ce que fait la fonction \inlinecode{generator}.}

\important{Pour les deux questions suivantes il faut que vos programmes fonctionnent même si la liste de tâches est modifiée.}

\exercice{Coder la fonction \inlinecode{scheduler} pour faire en sorte qu'elle affiche dans le terminal à chaque pas de temps (chaque message reçu de \inlinecode{generator}) la tâche qui devrait s'exécuter avec un ordonnacement srt (\emph{shortest remaining time}). Vous rendrez ce code sous le nom \file{schedv2.go}.}

% gp2, gp4 : sjf
% gp1 : fcfs
% gp3 : srt

\section{Client/serveur}

Le dossier \file{ex2} fourni avec cet énoncé contient du code source Go~: dans un dossier \file{server} vous trouverez le code d'un serveur très simple.

\exercice{Sur quel port le serveur codé dans \file{server/server.go} écoute-t-il ?}

\exercice{Parmi les fonctions utilisées par \file{server/server.go} lister celles pour lesquelles il aurait été nécessaire de traiter les erreurs potentielles.}

\exercice{Expliquer le principe général du modèle client/serveur en quelques mots (et éventuellement en vous appuyant sur un schéma).}

\exercice{Représenter graphiquement les messages qui devraient être échangés entre le client et le serveur pour faire afficher le message \emph{Bravo !} au serveur codé dans \file{server/server.go} dans le cas où toutes les valeurs tirées aléatoirement valent 0.}

\exercice{En vous basant sur la représentation graphique précédente des échanges entre client et serveur, codez un client qui fait afficher \emph{Bravo !} au serveur codé dans \file{server/server.go} quelles que soient les valeurs tirées aléatoirement. Vous rendrez votre code sous la forme d'un fichier \file{client.go}.}


\end{document}
